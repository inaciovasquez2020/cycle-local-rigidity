\documentclass[11pt]{article}

\usepackage[T1]{fontenc}
\usepackage{lmodern}
\usepackage{amsmath,amssymb,amsthm}
\usepackage{geometry}
\usepackage{hyperref}
\geometry{margin=1in}

% --- Theorem Environments ---
\theoremstyle{definition}
\newtheorem{definition}{Definition}[section]
\newtheorem{assumption}[definition]{Assumption}

\theoremstyle{plain}
\newtheorem{lemma}[definition]{Lemma}
\newtheorem{theorem}[definition]{Theorem}
\newtheorem{corollary}[definition]{Corollary}

\title{Structural Rigidity under Bounded Information Refinement}
\author{Inacio F. Vasquez \\ Independent Researcher}
\date{}

\begin{document}
\maketitle

\hrule
\vspace{0.5em}
\noindent
\textbf{STATUS:} VERIFIED / STRUCTURAL STATEMENT \\
\textbf{SCOPE:} Information-theoretic rigidity of refinement-based inference under locality, capacity, and extensivity constraints. \\
\textbf{DEPENDENCIES:} Shannon information theory; locality; thermodynamic scaling. \\
\textbf{NON-CLAIMS:} No algorithmic optimality; no convergence guarantees; no dynamics-specific results.
\vspace{0.5em}
\hrule

\section{Background}
Refinement-based inference procedures arise in measurement, control, and algorithmic reasoning. Such procedures operate by adaptively querying local observables and accumulating a transcript over time. When information gain per step is bounded, global reconstruction may be obstructed by structural limits independent of dynamics or geometry.

This manuscript isolates a minimal abstract spine capturing this obstruction.

\section{Motivation}
Under what conditions does bounded local refinement impose an unavoidable limit on globally extractable information?

\section{Intuition}
Each refinement step contributes only a bounded increment of mutual information. If the total information required by a task exceeds the cumulative capacity of the transcript, no adaptive strategy can succeed. This limitation persists regardless of computational power or dynamical sophistication.

\section{Local Distinguishability}
\begin{definition}[State Space]
Let $V$ be a finite vertex set and $\mathcal{H}=\bigotimes_{v\in V}\mathcal{H}_v$ a finite-dimensional Hilbert space. For $\Lambda\subseteq V$ and a density matrix $\rho$ on $\mathcal{H}$, define the reduced state
\[
\rho_\Lambda := \mathrm{Tr}_{V\setminus\Lambda}(\rho).
\]
\end{definition}

\begin{definition}[Local Distinguishability (LD)]
There exist constants $R<\infty$ and $\delta>0$ such that for all distinct sites $x\neq y$,
\[
\|\rho_{B_R(x)}-\rho_{B_R(y)}\|_1 \ge \delta .
\]
\end{definition}

\section{Bounded Information Refinement}
\begin{definition}[Transcript Model]
Let $X$ denote a global microstate. A refinement process produces a transcript $\tau_t=(O_1,\dots,O_t)$, where each $O_t$ is the outcome of an adaptive local query.
\end{definition}

\begin{assumption}[Per-step Capacity Bound (CB)]
There exists a constant $C<\infty$ such that for all $t\ge1$,
\[
I(X;O_t\mid \tau_{t-1}) \le C .
\]
\end{assumption}

\begin{definition}[Cumulative Refinement Capacity]
For $T\ge1$, define
\[
I_T := \sum_{t=1}^T I(X;O_t\mid \tau_{t-1}),
\qquad
I := \sup_{T\ge1} I_T .
\]
\end{definition}

\begin{lemma}[Alphabet Bound]
If each $O_t$ takes values in a finite alphabet $\Sigma$, then
\[
I(X;O_t\mid \tau_{t-1}) \le \log|\Sigma|.
\]
\end{lemma}

\begin{lemma}[Telescoping Bound]
For any $T\ge1$,
\[
I(X;\tau_T)=I_T \le T C .
\]
\end{lemma}

\section{Thermodynamic Extensivity}
\begin{assumption}[Amenable Geometry]
Let $(\Lambda_N)$ be an increasing exhaustion with $|\Lambda_N|\to\infty$ and finite-range Hamiltonians $H_{\Lambda_N}$. Assume the limit
\[
e^\ast := \lim_{N\to\infty}\frac{1}{|\Lambda_N|}\lambda_{\max}(H_{\Lambda_N})
\]
exists.
\end{assumption}

\begin{assumption}[Non-Amenable Geometry]
In a non-amenable setting, assume there exists a subsequence $(\Lambda_{N_k})$ such that
\[
e^\ast := \lim_{k\to\infty}\frac{1}{|\Lambda_{N_k}|}\lambda_{\max}(H_{\Lambda_{N_k}})
\]
exists.
\end{assumption}

\section{Structural Rigidity}
\begin{theorem}[Capacity Obstruction]
Under LD and CB, for any horizon $T$ and any derived observable $O=f(\tau_T)$,
\[
I(X;O) \le I(X;\tau_T) \le T C .
\]
\end{theorem}

\begin{corollary}[Infeasibility Beyond Demand]
If a task requires $I(X;O)\ge D$ for some $D>0$, then any refinement process with per-step capacity $C$ requires
\[
T \ge \frac{D}{C}.
\]
\end{corollary}

\section{Relationship Map}
This rigidity principle:
\begin{itemize}
\item isolates a universal information-theoretic obstruction,
\item is independent of dynamics and geometry,
\item serves as a reusable structural spine for domain-specific results.
\end{itemize}

\section*{References}
\begin{enumerate}
\item T.\ M.\ Cover and J.\ A.\ Thomas, \emph{Elements of Information Theory}, Wiley (2006).
\item R.\ Landauer, \emph{IBM J.\ Res.\ Dev.} 5 (1961).
\item B.\ Nachtergaele and R.\ Sims, \emph{Commun.\ Math.\ Phys.} 265 (2006).
\item O.\ Bratteli and D.\ W.\ Robinson, Springer (1987).
\end{enumerate}

\end{document}

