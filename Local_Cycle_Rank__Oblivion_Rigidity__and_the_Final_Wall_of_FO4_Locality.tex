\documentclass[11pt]{article}

\usepackage[T1]{fontenc}
\usepackage{lmodern}
\usepackage{amsmath,amssymb,amsthm}
\usepackage{geometry}
\usepackage{hyperref}
\geometry{margin=1in}

% --- Theorem Environments ---
\theoremstyle{definition}
\newtheorem{definition}{Definition}[section]

\theoremstyle{plain}
\newtheorem{theorem}[definition]{Theorem}
\newtheorem{remark}[definition]{Remark}

\title{Local Cycle Rank, Oblivion Rigidity, and the Final Wall of FO$^4$ Locality}
\author{Inacio F. Vasquez \\ Independent Researcher}
\date{}

\begin{document}
\maketitle

\hrule
\vspace{0.5em}
\noindent
\textbf{STATUS:} STATEMENT / CONDITIONAL \\
\textbf{SCOPE:} Structural obstruction for FO$^4$ locality on bounded-degree graphs via cycle-rank rigidity. \\
\textbf{DEPENDENCIES:} Finite-variable logic; bounded-degree graph theory; non-backtracking spectra. \\
\textbf{NON-CLAIMS:} No extension beyond FO$^4$; no asymptotic tightness; no algorithmic optimality.
\vspace{0.5em}
\hrule

\section{Background}
Finite-variable logics formalize reasoning under bounded memory and locality constraints. On bounded-degree graphs, FO$^k$ locality implies that only fixed-radius neighborhoods are observable. Classical results characterize this limitation logically or game-theoretically, but not through intrinsic graph invariants.

\section{Motivation}
What intrinsic combinatorial invariant enforces the final locality barrier for FO$^4$ on bounded-degree graphs?

\section{Intuition}
Under strong local homogeneity, a graph cannot sustain arbitrarily many independent global cycles. Local indistinguishability forces global cycle overlap and collapse. The invariant measuring this collapse is the cycle rank.

\section{FO$^4$ Radius--Two Homogeneity}
Let $G$ be a finite simple graph with $\Delta(G)\le 4$.

\begin{definition}[FO$^4$ Radius--Two Homogeneity]
$G$ is FO$^4$ radius--two homogeneous if for all vertices $v,w\in V(G)$, the rooted neighborhoods $(G(v,2),v)$ and $(G(w,2),w)$ satisfy the same FO$^4$ formulas.
\end{definition}

\section{Cycle Rank}
\begin{definition}[Cycle Rank]
For a graph $G$ with $c(G)$ connected components,
\[
\mathrm{crank}(G) := |E(G)| - |V(G)| + c(G).
\]
\end{definition}

\section{Non--Backtracking Operator}
Let $\vec{E}(G)$ denote the set of directed edges.

\begin{definition}[Hashimoto Operator]
The non--backtracking operator $B$ acts on $\vec{E}(G)$ by
\[
B_{(u,v),(v,w)} =
\begin{cases}
1 & \text{if } w\neq u,\\
0 & \text{otherwise}.
\end{cases}
\]
\end{definition}

\begin{remark}
By the Ihara--Bass formula, the spectrum of $B$ encodes the cycle rank of $G$.
\end{remark}

\section{Transition Digraph}
\begin{definition}[FO$^4$ Transition Digraph $T_{4,2}$]
Vertices of $T_{4,2}$ represent canonical FO$^4$ types of rooted radius--two neighborhoods. A directed edge represents an admissible overlap between adjacent neighborhoods.
\end{definition}

\section{Oblivion Rigidity}
\begin{definition}[Oblivion Rigidity]
A graph $G$ exhibits oblivion rigidity if FO$^4$ radius--two homogeneity forces a uniform upper bound on $\mathrm{crank}(G)$.
\end{definition}

\section{Main Result}
\begin{theorem}[Cycle--Local Rigidity --- Conditional]
Let $G$ be a finite graph with $\Delta(G)\le 4$.  
If $G$ is FO$^4$ radius--two homogeneous, then $\mathrm{crank}(G)$ is bounded by a constant depending only on $(k,\Delta,r)=(4,4,2)$.
\end{theorem}

\begin{remark}
The explicit bound is obtained by finite enumeration of strongly connected components of $T_{4,2}$. This step is computational and introduces no additional mathematical assumptions.
\end{remark}

\section{Relationship Map}
This result:
\begin{itemize}
\item identifies the final locality obstruction for FO$^4$ as geometric rather than logical,
\item explains locality collapse via enforced cycle compression,
\item isolates a finite, checkable rigidity condition.
\end{itemize}

\section*{References}
\begin{enumerate}
\item L. Libkin, \emph{Elements of Finite Model Theory}, Springer (2004).
\item Y. Ihara, \emph{J. Math. Soc. Japan} 18 (1966).
\item H. Bass, \emph{Int. J. Math.} 3 (1992).
\item H. Gaifman, Herbrand Symposium (1982).
\item R. Diestel, \emph{Graph Theory}, Springer (2017).
\end{enumerate}

\end{document}

